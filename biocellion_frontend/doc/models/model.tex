\subsection{\inlinecode{<model>}}

Every model must be contained in the \inlinecode{<model></model>} tag, with
exactly one model per XML file.  Each \inlinecode{<model>} may contain the following
tags.


\subsubsection{Allowed Subtags}

\begin{tabular}{ l | c | l}
  Tag & Required Count & Notes\\
  \hline
  \hline
  \inlinecode{<simulator>} & $1$ & \\
  \hline
  \inlinecode{<input>} & $0$ or $1$ & Not supported yet. \\
  \hline
  \inlinecode{<solute>} & $\ge 0$ & \\
  \hline
  \inlinecode{<molecule>} & $\ge 0$ & \\
  \hline
  \inlinecode{<particle>} & $\ge 0$ & \\
  \hline
  \inlinecode{<interaction>} & $\ge 0$ & \\
  \hline
  \inlinecode{<world>} & $1$ & \\
  \hline
  \inlinecode{<reaction>} & $\ge 0$ & \\
  \hline
  \inlinecode{<molecularReactions>} & $0$ & Not supported yet. \\
  \hline
  \inlinecode{<solver>} & $\ge 1$ & \\
  \hline
  \inlinecode{<agentGrid>} & $1$ & \\
  \hline
  \inlinecode{<species>} & $\ge 0$ & \\
\end{tabular}

\subsubsection{Allowed Attributes}

No attributes are allowed.

\subsubsection{Allowed Text Values}

No text values are allowed, only subtags.

\subsubsection{Example}

\begin{verbatim}
<model>
  ...  
</model>
\end{verbatim}
