\subsection{\inlinecode{<model>.<simulator>.<timeStep>}}

The \inlinecode{<simulator>} must contain one \inlinecode{<timeStep>} tag.
This tag is used to control the end of the simulation.

\subsubsection{Allowed Subtags}

\begin{tabular}{ l | c | l}
  Tag & Required Count & Notes\\
  \hline
  \hline
  \inlinecode{<param>} & & See required parameters below.\\
\end{tabular}

\subsubsection{Allowed Parameters}


\begin{tabular}{ l | c | c | c | p{1.5in} }
  Parameter & Required & Type & Units & Notes \\
  \hline
  \hline
  \inlinecode{endOfSimulation} & Yes & float & time & Simulated time to end the simulation. \\
  \hline
  \inlinecode{adaptive} & No & boolean & & Not supported yet. \\
  \hline
  \inlinecode{timeStepIni} & No & float & time & Not supported yet. \\
  \hline
  \inlinecode{timeStepMin} & No & float & time & Not supported yet. \\
  \hline
  \inlinecode{timeStepMax} & No & float & time & Not supported yet. \\
\end{tabular}


\subsubsection{Allowed Attributes}

No attributes are allowed.

\subsubsection{Allowed Text Values}

No text values are allowed, only subtags.

\subsubsection{Example}

\begin{verbatim}
<model>
  ...  
  <simulator>
    ...  
    <timeStep>
      <param name="endOfSimulation" unit="hour">1.0</param>
    </timeStep>
  </simulator>
  ...  
</model>
\end{verbatim}
