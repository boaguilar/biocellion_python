\subsection{\inlinecode{<simulator>}}


Every \inlinecode{<model>} must contain one \inlinecode{<simulator>} tag.
This tag is used to control high-level time related settings for the simulation.

\subsubsection{Allowed Subtags}

\begin{tabular}{ l | c | l}
  Tag & Required Count & Notes\\
  \hline
  \hline
  \inlinecode{<param>} & & See required parameters below.\\
  \hline
  \inlinecode{<timeStep>} & 1 & \\
\end{tabular}

\subsubsection{Allowed Parameters}


\begin{tabular}{ l | c | c | c | p{1.5in} }
  Parameter & Required & Type & Units & Notes \\
  \hline
  \hline
  \inlinecode{outputPeriod} & Yes & float & time & Simulated time between data output. \\
  \hline
  \inlinecode{agentTimeStep} & Yes & float & time & Simulated time delta per baseline timestep \\
  \hline
  \inlinecode{numStateAndGridTimeStepsPerBaseline} & Yes & integer &  & Internal timesteps per baseline timestep. \\
  \hline
  \inlinecode{restartPreviousRun} & No & boolean & & Not supported yet. \\
  \hline
  \inlinecode{randomSeed} & No & integer & & Not supported yet. \\
  \hline
  \inlinecode{chemostat} & No & boolean & & Not supported yet. \\
  \hline
  \inlinecode{diffusionReactionOnAgentTime} & No & boolean & & Not supported yet. \\
\end{tabular}


\subsubsection{Allowed Attributes}

No attributes are allowed.

\subsubsection{Allowed Text Values}

No text values are allowed, only subtags.

\subsubsection{Example}

\begin{verbatim}
<model>
  ...  
  <simulator>
    <param name="outputPeriod" unit="hour">0.1</param>
    <param name="agentTimeStep" unit="hour">0.1</param>
    <param name="numStateAndGridTimeStepsPerBaseline">1</param>
    ...  
  </simulator>
  ...  
</model>
\end{verbatim}
