\subsection{\inlinecode{<param>}}

\inlinecode{<param>} tags are used for specifying various parameters
to define the model.  For example, how many simulated hours to
run the simulation.
Parameters have a name attribute, and a text value.  Most parameters
also expect a unit attribute.  Many unit conversions are supported.

\subsubsection{Allowed Subtags}

No subtags are allowed.

\subsubsection{Allowed Parameters}


No parameters are allowed.



\subsubsection{Allowed Attributes}

\begin{tabular}{ l | c | p{1.5in} }
  Attribute & Required & Notes \\
  \hline
  \hline
  \inlinecode{name} & Yes & Name of the parameter. \\
  \hline
  \inlinecode{unit} & Recommended & Units of the value. \\
\end{tabular}

\subsubsection{Allowed Text Values}

The text values is usually a floating point number, but could be
an integer, or a boolean specification.

\subsubsection{Examples}

\begin{verbatim}
<model>
  ...  
  <simulator>
    <param name="outputPeriod" unit="hour">0.1</param>
    <param name="agentTimeStep" unit="hour">0.1</param>
    <param name="numStateAndGridTimeStepsPerBaseline">1</param>
    ...  
  </simulator>
  ...  
</model>
\end{verbatim}
